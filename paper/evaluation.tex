\section{Evaluation} \label{section:evaluation}

\subsection{Client Complexity}

\subsection{Performance}

\subsection{Fault Tolerance}

\subsection{Challenges}

\todo{rephrase this so it flows}
Our current implementation restricts values to be strings to simplify development. 
Handling typing with Go's \texttt{interface\{\}} type is slow because of runtime reflection.
Because Go is garbage collected, a large numbers of heap allocations can incur high latencies during
operation. Many protocol encoder/decoders in Go create many temporary objects, which, due to Go's poor escape analysis, get promoted to heap allocations. 

This issue can be addressed with a number of techniques. 
Firstly, using typed formats such as MsgPack, CapnProto and Protobuf allows parsing code to ``plan-ahead'' for which types to use and how much space they will use. 
JSON is particularly bad at this; because it is a character-based format, parsing it requires many intermediate buffers and lookaheads to determine the size and type of elements in a received message.
Secondly, using generated code for encoding/decoding can further reduce the number of allocations. We use the excellent \texttt{msgp} library\footnote{https://github.com/tinylib/msgp}, which reduced the allocations per message from roughly 20 to 3.
