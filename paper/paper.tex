\documentclass[conference]{IEEEtran}
\usepackage{enumitem}
\usepackage{graphicx}
\usepackage{url}
\usepackage{xcolor}
\usepackage{listings}
\usepackage[labelfont=bf,format=plain,font=small]{subcaption}
\usepackage[labelfont=bf,format=plain,font=small]{caption}
\usepackage{array}
\newcolumntype{L}{>{\arraybackslash}m{3cm}}
\usepackage{minted}
\newminted{sql}{mathescape, numbersep=5pt, frame=lines, framesep=2mm, fontsize=\footnotesize}
\usepackage{MnSymbol}
\def\prebreak{\raisebox{0ex}[0ex][0ex]{\ensuremath{\rhookswarrow}}}
\def\postbreak{\raisebox{0ex}[0ex][0ex]{\ensuremath{\hookrightarrow\space}}}
\def\lstbreak{\prebreak\newline\postbreak}
\lstdefinelanguage{pseudocode}{sensitive=false,morecomment=[l]{//},morestring=[b]",
                               morekeywords={if,else,while}}
\lstset{breaklines=true,breakindent=5pt,
        postbreak=\postbreak,prebreak=\prebreak}

\newcommand{\todo}[1]{\textcolor{red}{TODO: #1}\PackageWarning{TODO:}{#1!}}


\begin{document}

% TODO playing with new ways of wording the title; the content we had was correct
% but I didn't feel great about the wording (and don't feel great about this either)
\title{Continuous Query-Based Syndication in the Cloud for Scalability and Availability}

\author{\IEEEauthorblockN{Gabriel Fierro}
\IEEEauthorblockA{gt.fierro@berkeley.edu}
\and
\IEEEauthorblockN{Erik Krogen}
\IEEEauthorblockA{erikkrogen@berkeley.edu}
}

\maketitle

\begin{abstract}
TODO Abstract
\end{abstract}

\section{Introduction}
% just copying over chunks of the old intro
Applications in the Internet of Things exist at a confluence of semantically isolated networks over buildings, physical spaces, cloud services, smart appliances and mobile and wearable devices. 
The utility of these applications is in their ability to capture capabilities of new families of smart, networked devices and integrate them with existing networks surrounding people, things and places.

Most modern systems that address composition of services over networked things deal primarily with hardware abstraction -- the need to separate \emph{what} is being sensed or actuated from \emph{how} that action is performed -- and syntactic interoperability -- common packet formats and canonical descriptions of common capabilities. 
These systems (including CORBA~\cite{vinoski1997corba}, Jini~\cite{waldo1999jini}, AllJoyn~\cite{alljoyn}, IoTivity~\cite{iotivity}) generally offer limited discovery capabilities that only identify implementers of a given interface, and not how that implementer is related to other resources required by the application. 
In other words, this approach assumes that the application or application developer has enough contextual information on the set of discovered resources to disambiguate which are relevant to the application. 
The notion of relevance extends beyond what a resource logically represents (a temperature sensor, a light bulb, etc) and includes location, ownership, and how that resource links to other entities such as in a building management system.

The current approach is insufficient for establishing meaningful collections of devices and services because it assumes that either a) the domain of discovery is small enough to remove any ambiguities on the identity of returned devices, or b) the application or user possesses prior contextual information on those devices.  
This ``out-of-band'' information becomes harder to manage from application and user perspectives as deployments get larger and devices get more diverse.

In this paper, we present the design and implementation of a distributed CQBS broker, a discovery service and message broker for service composition that uses \emph{continuous query-based syndication} to enable dynamic and contextually-aware applications and services for the Internet of Things. 

We begin by outlining the core
primitives of CQBS, \emph{streams}, which are virtual representations of
specific sensor and actuator capabilities.  
We then define the syndication
model, which describes relationships between streams to create ad-hoc
collections of capabilities needed for applications. 
Syndication queries are
expressed using SQL-like relations over stream descriptions and are
continuously evaluated to maintain consistency with the environment.  
\todo{also mention how we're going to evaluate it}
\todo{want to explicitly mention our goals here}
% also want to capture how expressive it is
% do we mention here who we compare against?

 % for later

% TODO do we need a motivation section or is it included in introduction?
% I'm thinking related work up front since it ties into motivation pretty well 

\section{Motivation} 

\section{Related Work}
\label{section:relatedwork}

The CQBS distributed broker is most closely related to \emph{publish-subscribe} systems.
We examine each of the systems below along the primary design dimensions of the CQBS broker: richness of publisher descriptions, expressiveness of syndication model, client complexity and fault tolerance.
A summary of these systems and their properties can be seen in Table~\ref{table:comparison}.

\begin{table*}[h]
\caption{High-level comparison of features between systems}
\label{table:comparison}
\centering
\begin{tabular}{|l|c|c|c|c|c|}
\hline
\textbf{System Name} & \textbf{Publisher Descriptions} & \textbf{Syndication} & \textbf{Client Complexity} & \textbf{Client Failover} & \textbf{Fault Tolerant} \\
\hline \hline
redis~\cite{redis} & $N$ channels & List of channel names & Simple & None & Replicated cluster \\
MQTT~\cite{locke2010mq}\cite{hunkeler2008mqtt} & Hierarchical topics & Wildcard matching & Simple & None & None \\
XMPP~\cite{saint2011extensible} & Unique names & List of publishers & Complex & None & Federated Servers \\
Kafka~\cite{kreps2011kafka} & Hierarchical topics & Wildcard matching & Complex & Yes & Replicated broker cluster \\
SIENA~\cite{carzaniga2000achieving} & attribute-value pairs & SQL-like predicate & N/A & N/A & Replicated broker, flexible routing \\
JMS~\cite{hapner2002java} & attribute-value pairs & SQL-like predicate & Complex & Unknown & Yes \\
CORBA~\cite{vinoski1997corba} & properties and names & property matching & Complex & No & No \\
\textbf{CQBS Broker} & attribute-value pairs & SQL-like predicate & Simple & Yes & Replicated brokers, coordinators \\
\hline
\end{tabular}
\end{table*}

\subsection{Topic-Based Publish-Subscribe}

Most publish-subscribe (``pub-sub'') systems fall into one of two categories: topic-based and content based~\cite{eugster2003many}.
The most basic form of topic-based pub-sub is a channel model, in which producers (data publishers) transmit data associated with some channel name to a broker; subscribers list the channels in which they are interested.
The benefits of this approach are its simplicity and speed---the ``hot path'' of a published message simply retrieves a list of subscribers---but each publisher is limited in expressive power to a single dimension (the name of the channel)~\cite{redis}.
Because of these limitations, some modern pub-sub systems use hierarchical topics with prefix and suffix matching using wildcards. The most popular of these are MQTT~\cite{locke2010mq}, Kafka~\cite{kreps2011kafka} and XMPP~\cite{saint2011extensible}.
%While our CQBS system is primarily a \emph{publish-subscribe} solution, its goals of providing
%resource discovery as well as data transfer means that we must also

%We examine each of the systems below along the primary design dimensions of
%the CQBS broker: richness of publisher descriptions, expressiveness of
%syndication model, client complexity and fault tolerance.

\textbf{MQTT} is a lightweight publish-subscribe protocol popular for its simpicity and extensible messages.
Producers of data in MQTT publish on hierarchical path-like topics such as \texttt{/apartment/gabe/livingroom/temperature}.
This construction of topics is best for grouping publishers together along a limited number of dimensions, but quickly becomes unwieldy as the dimensionality and sparsity of the descriptions increase.
For example, a temperature sensor could be described along the following dimensions:

\begin{itemize}
\item manufacturer and model number
\item city, campus, building, floor and  room number
\item orientation or position within a room
\item accuracy and precision of temperature sensor
\item method of temperature sensing (e.g. IR, thermopile)
\item who installed the temperature sensor and when it was installed
\item sample rate of the temperature sensor
\end{itemize}

To be effective, a topic-based system must determine the order and syntax of topics so that subscribers can know they are consuming the appropriate streams.
This is further complicated when considering other types of publishers which may include an entirely different set of descriptive tags.
MQTT syndication supports prefix matching on topics and limited forms of suffix matching.
To subscribe, applications specify explicit topics (\texttt{a/b/c/d}), one-level wildcards (\texttt{+/b/c/d, a/+/c/d, a/+/+/d, a/b/c/+}) and multi-level wildcards (\texttt{\#, a/\#, +/b/c/\#}).
While appropriate for basic subscriptions, this approach does not allow the expression of more complex predicates that contain ``and'', ``or'' or ``not'' relations: temperature sensors in Gabe's apartment, but not the ones in the kitchen or the bedroom.

MQTT supports three Quality-of-Service levels for message delivery: at most once, at least once, and exactly once.
Most client implementations simply address the first two, keeping complexity and code-size down; it is entirely feasible to implement a MQTT client on an embedded device, and there exists an adaptation of MQTT (MQTT-S~\cite{hunkeler2008mqtt}) for non-TCP/IP networks.
MQTT does not contain any explicit fault tolerance mechanisms: brokers may be distributed and ``bridged'' with the aid of an administrator, but failover logic is entirely implementation dependent.
In summary, MQTT is insufficient for our goals because hierarchical topics are fundamentally constraining in their structure and syndication.

\textbf{XMPP}~\cite{saint2011extensible}, or the Extensible Messaging and Presence Protocol, is an XML-based technology for instant messaging, video conferencing and more recently sensor communication (as seen in large, deployed systems such as Sensor Andrew~\cite{rowe2011sensor}).
Similar to channel-based systems, every entity (publisher or subscriber or broker) in a distribution of federated XMPP servers has some unique address from which it can send and receive messages.
There are attempts to provide a more expressive pub-sub model on top of XMPP that allows for the discovery of services and subsequent subscription to relevant service providers~\cite{millard2010xep}.
These service descriptions can be extended to include arbitrary contextual data, which is a definite advantage over primitive address-based messaging.

The limitation of XMPP lies in its size and complexity.
XMPP messages are XML-encoded and thus permit the expression of many different structures, but XML parsers are typically large and memory-intensive, rendering them inappropriate for embedded devices.
Additionally, XML tends towards large messages, which can result in fragmented messages in embedded networks that lower throughput and delivery rate.

\textbf{Apache Kafka}~\cite{kreps2011kafka} is a distributed messaging system designed for data pipelining in large, distributed, high-throughput applications.
Kafka is not designed for deployment scenarios that require rich descriptions of the array of available data services, so publisher and subscriber interactions are done via hierarchical topics and wildcard matching (very similar to MQTT).

Kafka, unlike MQTT and XMPP, is designed to be fault tolerant: all topics are replicated in a cluster of brokers, and failover is automatic.
To achieve the combination of fault tolerance and performance, Kafka clients must be carefully engineered, and are intended to be fuller applications rather than embedded devices.
Thus, while Kafka may be suitable for a data anaylsis pipeline in a datacenter, it does not meet our requirements for message delivery and reception at the ``edge'' of an IoT network.

\subsection{Content-Based Publish-Subscribe}

In content-based pub-sub systems, publishers attach richer descriptions to messages, allowing subscribers to specify predicates that act as filters for which messages they receive.
While this scheme has richer descriptive power, routing on a per-message basis is computationally expensive (reducing routing efficiency) and can require larger messages from publishers.
% these will be fairly short

\textbf{SIENA}~\cite{carzaniga2000achieving}, the Scalable Internet Event Notification Architecture, aims to maximize the expressiveness of publishers and subscribers communicating in a wide-area network.
SIENA publishers send messages containing a set of \texttt{<type, name, value>} tuples, to which clients can subscribe using SQL-like filter expressions.
This approach, designed to provide discovery of relevant data in a large number of heterogeneous messages, allows publishers to express richer metadata than would be feasible using a topic-based scheme.
It also allows subscribers to more precisely define the data they need.
For example, this could easily capture the proposed descriptive elements of the temperature sensor described above and allow a subscriber to filter on any combination of those attributes.

SIENA is fully distributed, using a tree overlay for routing and a special ``merging'' mechanism for pruning unnecessary delivery of messages to subtrees.
The disadvantages of SIENA mirror those of other content-based systems: carrying a full description of a publisher with every message reduces the bytes available for application data in embedded, constrained networks typical of the IoT.

\textbf{JMS}~\cite{hapner2002java}, the Java Message Service, is a distributed messaging service for connecting distributed application components.
Publishers send messages with headers containing standardized key-value pairs, but also contain lists of user-defined key-value properties.
Like SIENA, JMS clients subscribe with SQL-like expressions that express constraints on the set of publisher properties.
One difference between JMS and other systems is its default setting of exactly-once delivery, which complicates client logic and raises the network overhead of sending or receiving a message.

Distribution and fault-tolerance in JMS is possible, but requires specific configuration of which topics are distributed and among which servers they are distributed.
The focus of JMS is on enterprise-type applications that are in need of a messaging system that can adapt to its needs, but does not need to do so dynamically.

%\textbf{Elvin}

%\subsection{Other Systems}

%\textbf{Tuple Spaces} \todo{tuple spaces: t-spaces and linda}

\textbf{CORBA}~\cite{vinoski1997corba}, the Common Object Request Broker Architecture, is a data bus for communication amongst distributed objects that provides property- and name-based discovery and event notification.
Distributed object models (including DCOM~\cite{horstmann1997dcom}) share many features with content-based pub-sub systems: similar to JMS, published messages contain key-value pairs in the header and body, and syndication is performed by subscribers specifying filters on those attributes.
CORBA itself is limited because it is not designed to be distributed, and has no failover or replication mechanisms.

%%% Local Variables:
%%% mode: latex
%%% TeX-master: "paper"
%%% End:
 % Gabe

\section{Continuous Query-Based Syndication}

Continuous Query-Based Syndication (CQBS) is a hybrid publish-subscribe pattern that provides the expressiveness of content-based systems while retaining the simplicity of topic-based systems.
The goal of CQBS is to provide a messaging system that can account for and adapt to the heterogeneity of data sources in the IoT.
CQBS endows embedded publishers to describe themselves using rich metadata, and provides subscribers to discover and subscribe to relevant data sources and maintain a consistent view of the context of those sources.

Here, we first establish the CQBS primitives --- streams and metadata --- before delving into how CQBS operates and what roles publishers and subscribers play.
We then describe the design and implementation of an individual broker (we defer the discussion of the full distributed system to Section~\ref{section:coordinator}.

\begin{figure}
\centering
\begin{lstlisting}[language=pseudocode,basicstyle=\small]
UUID = "dd9ef92e-140a-11e6-b352-1002b58053c7"
# register client
metadata = {
  UUID = UUID,
  Location/Room = "410",
  Location/Building = "Soda",
  Location/City = "Berkeley",
  Point/Type = "Sensor",
  Point/Measure = "Temperature"
  UnitofMeasure = "Fahrenheit",
  UnitofTime = "ms",
  Timezone = "America/Los_Angeles"
}
register_msg := msgpack.encode(metadata)
send_to_broker(register_msg)
while True:
    temp_val := read_sensor()
    msg := msgpack.encode({
            UUID = UUID,
            Value = temp_val
           })
    send_to_broker(msg)
    sleep(10)
\end{lstlisting}
\caption{Client registration and publishing pseudocode}
\label{fig:pseudoclient}
\end{figure}

\subsection{Streams and Metadata}

A stream is a virtual representation of a specific sensor or actuator channel (a ``capability'') that is indexed by a 16-byte universally unique identifier (UUID).
Each stream is described by \emph{metadata}, which is a bag of key-value pairs: keys are required to be string, but values may be any one-dimensional data type\footnote{In our implementation, values are restricted to strings: see Section~\ref{section:evaluation}.}.
Key-value pairs are most effective when drawn from some well-known ontology (such as Semantic Sensor Web~\cite{sheth2008semantic}), but our system places no restrictions on their content.
The association of metadata to a stream is done by the UUID; when a publisher creates a new stream, it registers that stream with the broker by sending a message containing the UUID and all of the metadata.
The broker (or the coordinator, in the distributed case) stores the mapping from stream UUID to metadata.
A publisher changes metadata by sending the ``diff'' of which keys and values have changed.
A given producer (data provider) can have as many streams as it wishes.
Each message contains at least the UUID of the originating stream, and can also contain any metadata changes, and of couse the published value itself, which can be any serializable object.

An example of metadata for a temperature sensor, and the basic client logic, can be found in Figure~\ref{fig:pseudoclient}.
In the initial registration message, along with the other metadata, the reporting process describes the thermostat as being in room 410 Soda.
If this changes, such as if the sensor were on a piece of smart clothing or furniture, the sensor attaches the metadata update \lstinline{Location/Room = "415"} to any outgoing message, where it is handled by the broker (described in the following section).
A discussion of client complexity can be found in Section~\ref{section:evaluation}.

This is a departure from the approach of content-based pubsub systems, where although the producer may possess some unique identifier, it transmits any associated ``content'' (metadata) in every message.
This verbose design choice may be appropriate for distributed sytems in which a publisher is a larger application that produces many different types of data, but when the data per-producer is relatively static (temperature sensors will always report temperature data), this flexibility is unneeded.
It becomes more efficient to essentially ``cache'' the metadata of a publisher in a central location where it can be used for syndication.

The simple structure of a stream (essentially a set of special key-value pairs) means a stream can be well represented in nearly any application protocol.
We choose MsgPack, a lean, typed binary serialization format that is simple enough to be encoded/decoded on embedded devices with limited code space.

%\subsection{Clients}

%Clients are distributed, continuously running processes that are producers and consumers of streams and updates to metadata.
%The streams produced by clients represent the set of sensors, actuators and capabilities available to be discovered, consumed and used.

% \emph{message} is the unit of communication in all interactions between
%clients and the broker. Messages sent from clients are either a query string
%for initiating a subscription or one-off response, or a structured packet
%containing the UUID of some stream and optionally an array of readings and/or a
%set of metadata updates. Examples of when metadata updates occur include when a
%device is registered (i.e.  sending the initial configuration), when a device
%changes location, or when a equipment attached to a device is changed such as
%installing an occupancy-driven switch for a lighting system.  Messages sent by
%the broker consist of forwarded client messages and ``diffs'' that convey
%changes in the set of streams matching a client's query.
%Figure~\ref{fig:messages} contains an example exchange of messages for
%a continuous query.
%
%Upon startup, a client registers streams it produces by sending messages
%containing metadata describing each one (see Figure~\ref{fig:message}). This makes
%the stream discoverable by and visible to other clients. Reporting data is
%straightforward: clients simply send messages with a stream's identifier and
%latest timestamped readings to the broker (the \texttt{UUID} and
%\texttt{Readings} fields). The broker forwards these readings to any subscribed
%clients as it receives them.
%
%In addition to producing streams, clients can also consume streams by
%expressing subscriptions to the broker in the form of SQL-like queries (see
%Section~\ref{section:syndication}). These queries are how clients perform
%discovery, receive timeseries data, and receive actuation requests.
%In dynamic environments typical of the Internet of Things, the results of
%these queries can become stale if they are executed only once. The broker
%continuously evaluates these queries to provide clients with a consistent view
%of their operating context.
%
%\subsection{Giles Architecture} \label{subsection:architecture}
%
%In this section, we present an overview of the six components of
%the Giles architecture.  These components handle incoming messages and queries
%for archival and delivery, and maintain consistent routes for query-based
%syndication.  This architecture is illustrated in
%Figure~\ref{fig:architecture}.
%
%\textbf{Protocol Plugin}: Protocol plugins do the work of translating messages
%between an application protocol and canonical message form. For
%simple request-response cases, the plugin translates the request, relays
%the message to the Giles API, then converts the result back to the original
%application protocol format and responds to the client. For the streaming
%functionality required by query-based syndication, the plugin maintains the
%connection to each client, and provides a handler to the Giles API that is
%called whenever an event is to be sent to an associated client.
%
%\textbf{API}: protocol plugins interface express how an incoming
%message should be interpreted. In terms of incoming data, Giles only deals with
%queries and messages (updates on streams involving metadata and/or new
%timeseries data), but can process them differently depending on the context.
%The API hands the incoming data to the correct pipeline.
%
%
%\textbf{Authorization Manager}: Giles enforces read/write permissions on a
%stream by stream basis. Considering the large numbers of devices and contexts
%predicted in the Internet of Things, it is important that
%administration of the system needs to scale, not just the system itself. Giles
%uses group-based permissions, rather than individual permissions, to avoid
%micromanagement of permissions surrounding increasing numbers of applications and users. All individual
%interactions with Giles carry an \emph{ephemeral key} which has an expiry and
%can be revoked; this key maps onto a set of roles. For each incoming
%interaction, the authorization manager does the work of evaluating the key's
%associated roles with the streams involved in the interaction, and adjusts the
%result set appropriately.
%
%\textbf{Broker}: While protocol plugins handle the network element of
%publishing and subscribing, the broker component provides routing
%between publishers and subscribers. This is done by maintaining mappings between clients, their associated
%queries, and the streams associated with those queries. These mappings are kept
%consistent in-band with the rest of the system, as discussed in
%Section~\ref{subsection:eventdriven} and seen in Figure~\ref{fig:evaluatequery}.
%
%\textbf{Query Processor}: The query processor parses all incoming queries and
%uses the outputted abstract syntax tree (AST) to adjust broker data structures
%(detailed in Figure~\ref{fig:evaluatequery}). This component also handles
%communication to and from the metadata and timeseries databases, and
%reevaluates subscription queries on behalf of the broker.
%
%\textbf{Timeseries Database}: Giles' role as a broker makes it a logical
%location to do archival of incoming data. Even in a distributed setting with
%multiple broker instances, Giles' visibility into all routed data means that
%archival can be performed without explicit action by clients. This is in
%contrast to other service composition systems (e.g. AllJoyn~\cite{alljoyn},
%SDS~\cite{czerwinski1999architecture},
%Jini~\cite{gupta2002jini}\cite{rigole2002using}\cite{waldo1999jini}) that
%perform archival as a secondary feature. All participants in such a system must
%duplicate all data to be archived and communicate that directly with an
%archival service. In Giles, all data and interactions are archived
%automatically and can be queried. Giles stores timeseries data in a dedicated
%timeseries store that is updated dynamically as clients report.
%
%\textbf{Metadata Database}: The Giles metadata model is simple enough to not
%place unusual requirements on the choice of a backend database.  Streams are
%wholly defined by their UUID and metadata, so the database only needs to store
%associations of UUID to metadata, and provide query facilities for retrieving
%metadata for a given UUID and retrieving all UUIDs whose keys fulfill a
%SQL-like predicate.  Stream metadata changes over time, which may invalidate
%any attempted categorization of streams. This makes the application of a strict
%relational schema unwieldy for maintaining the association of UUID to metadata,
%and informs our decision to impose structure over metadata using client-defined
%queries.
%
\subsection{Query-Based Syndication}

A primary contribution of our distributed broker architecture is its continuous, query-based syndication.
Queries are structured, SQL-like statements that define sets of constraints over stream metadata to express ad-hoc relationships between streams.
Query-based syndication is the use of these queries to define the forwarding routes from publishers to subscribers.
The broker persists bindings of streams to clients formed by evaluating subscription queries against the set of available metadata, and routes incoming messages according to those bindings.
The resolution of queries to routes is \emph{continuous}; the broker reevaluates syndication queries as stream metadata evolves, and informs clients of changes in the set of the streams to which they are subscribed.
These changes happen on any metadata event: stream registration, stream deletion and metadata updates on streams.

%Queries are simple strings that support following operators (from giles). Examples of queries include blah blah blah.
%figure messages illustrates...

\begin{figure*}[t]
\centering
\includegraphics[width=.8\linewidth]{figs/messages.pdf}
\caption{The network traffic for a continuous query for discovering all temperature sensors in room 410 (ommitted
for brevity are additional constrains for building, units, etc). As new streams are registered, or their metadata
changes to no longer fit the discovery constraints, the client is updated in real-time.}
\label{fig:messages}
\end{figure*}


\section{Design \& Architecture} % ETK

% TODO ETK make sure 'publisher', 'subscriber', 'client' have all been defined by this point
% TODO ETK if these goals are already outlined somewhere else then no need to put them here

Our design is motivated by a number of goals that we wish to achieve:
\begin{itemize}
\item High Availability: We wish to create a system that is resilient in the face of arbitrary machine failures.
\item Scalability: The system should be able to scale to large numbers of clients with reasonably high message rates.
\item Simple Clients: The code necessary for a client to interact with the system should be very simple, since we assume that they may be embedded devices with limited programming facilities.
\end{itemize}

\subsection{Overview}

\begin{figure*}[t]
\centering
\includegraphics[width=6.5in]{figs/full_architecture.pdf}
\caption{Overview of the architecture of our brokerage system.
Numerous clients communicate to a set of decentralized brokers which create a forwarding network between themselves as instructed by the centralized coordinator nodes.}
\label{fig:architecture}
\end{figure*}

To meet these goals, we have developed an architecture which consists of two primary components: distributed brokers and centralized coordinators.
See Figure~\ref{fig:architecture} for an overview which will be described in more detail in this and the following sections.

The system contains one logically centralized coordinator; to all other entities in the system, the coordinator can be treated as a single machine.
In actuality, this logical coordinator consists of three independent nodes to improve fault tolerance; see Section~\ref{subsec:coordinator_fault_tolerance} for more detail.
This coordinator makes all of the decisions in the system, determining when a broker has failed, which brokers should forward which messages where, when changes occur to the set of publishers a subscriber is currently receiving messages from, which broker a client should contact if their broker fails, etc.
It then distributes these decisions to the brokers, which take appropriate action.
To do this the coordinator stores the current state of all brokers in the system, as well as information about all of the clients that are known to the system, i.e.\ what broker they are attached to, what query they are interested in (for subscribers), and what their current metadata is (for publishers).

Brokers are numerous and may reside anywhere; for example, a deployment may consist of a broker located in each building which contains client devices, or brokers may be run on cloud computing nodes.
Brokers are responsible for communicating with clients and for forwarding messages along routes as instructed by the coordinator.
Any changes to the set of clients connected to the broker, or to the metadata of a publisher connected to the broker, are communicated back to the coordinator for handling so that the coordinator always has an update-to-date view of the entire system state.

\subsection{Normal Operation}

In this section we describe the events which take place under normal operation, i.e.\ in the case that there are no failures within the system.

\textbf{A new subscriber submits a query.}
A subscriber contacts its local broker, whose address can be hardcoded into the client or discovered through some network discovery protocol, e.g.\ % TODO Gabe can you help here
The subscriber submits a message containing the query for which it is interested in receiving relevant messages.


\subsection{Broker Fault-Tolerance}

\subsection{Coordinator Fault-Tolerance}
\label{subsec:coordinator_fault_tolerance}

\subsection{Design Discussion}

% Evaluate design: simplicity of client protocol, etc. etc.

Note that this design creates a bottleneck at the coordinator, which must be contacted for all changes to the state of the system.
We assume that in comparison to the rate of messages being sent in the system, changes to the set of connected clients and to the metadata associated with publishers is relatively slow.
Since message forwarding can be done locally at a broker, distributing the brokers allows for high scalability in terms of number of messages that can be sent in the system, though if changes to the system state occur frequently enough the coordinator will become overloaded.
Part of the reason for choosing this design was its relative simplicity; we have two alternate designs discussed in Section~\ref{subsec:alternate_designs} which were considered and which we hope to evaluate as future work.


\subsection{Alternate Designs Considered}
\label{subsec:alternate_designs}
?


\section{Evaluation}
\label{section:evaluation}

Here we present the evaluation of a distributed CQBS broker.
All brokers and coordinators machines were chosen to emulate commodity hardware; for CQBS to be an effective solution, it must not rely on intractably large resources.
Hence, we chose to use the \texttt{t2.medium} AWS instance type with 2 vCPUs and 4 GB RAM, running Ubuntu 14.04.
As we will demonstrate below, the CQBS system is amenable to commodity systems because its performance is limited by the serialization of the etcd log, and not by memory, CPU, disk, or network bandwidth.

\subsection{Broker Performance}

\begin{figure}[t]
\centering
\includegraphics[width=\linewidth]{figs/singlenodelatency.png}
\caption{Microbenchmark: standalone CQBS broker forwarding latency with increasing concurrency}
\label{fig:singlenodelatency}
\end{figure}

First, we examine the latencies of the CQBS broker's forwarding mechanism in isolation---without the communication overhead imposed by the fully replicated system.
We run a single broker on a \texttt{t2.medium} instance, backed by MongoDB\@.
Using a ratio of 10 publishers to one subscriber, we run three benchmarks with 1, 10 and 100 subscribers (with 10, 100 and 1,000 publishers accordingly).
Each publisher sends 5 messages per second; after the initial registration message (the high latencies at the beginning of each benchmark), each publisher sends only its stream UUID and an increasing counter as its value.
Each group of publishers/subscribers use entirely isolated sets of keys, so that they do not explicitly interfere with each other in the broker.
These three microbenchmarks are shown in Figure~\ref{fig:singlenodelatency}, and demonstrate that the latency is fairly consistent as the amount of concurrency scales.

The spikes in latency seen in the $N=1,10$ graphs are due to the pauses enacted by Go's garbage collection.
Fortunately, these do not affect the vast majority of requests.
For the $N=1,10$ cases, the mean latencies, 95\textsuperscript{th} percentile latencies and standard deviation are $4.67ms/4.85ms/0.55ms$ and $4.62ms/4.77ms/0.37ms$ respectively.
For the $N=100$ case, the garbage collection becomes more visible in the variability of response times, with mean, 95\textsuperscript{th} percentile and standard deviation latencies of $13.58ms/8.87ms/67.13ms$.
The troughs in the $N=100$ case are an odd phenomenon, most likely caused by garbage collection in the single Go process used to generate the 100 subscribers and 1000 publishers, generating approximately 5000 messages per second.

\subsection{Coordinator Performance}

\todo{include existing graph}

\subsection{Full-System Performance}

\todo{include existing graph}

\subsection{Fault Tolerance}

\todo{fault tolerance metrics when a broker dies}

\subsection{Client Complexity}

One goal of this system was to maintain a low level of client complexity.
To evaluate this, we have written clients in the Go and Python languages.
Publishers are capable of publishing values and modifying their metadata, and subscribers are capable of submitting queries and attacking handler functions to respond to inbound published messages and notifications about changes to the set of publishers which they are currently subscribed to.
Both types of clients are capable of contacting the coordinator to seamlessly handle the failure of their local broker.

\begin{table}
\centering
\caption{Lines of non-comment, non-whitespace code used to implement programmable clients in Python and Go.
Base Code is the basic code necessary to communicate with the system, Failover is the code necessary to communicate with the coordinator to handle broker failures, and Subscriber/Publisher are the code necessary to implement subscriber- and publisher-specific functionality on top of the shared code.}
\label{tbl:client_code}
\begin{tabular}{ | c | c | c | c | c | c | }
\hline
Language & Base Code & Failover & Subscriber & Publisher & Total
\\\hline
Go & 175 & 111 & 65 & 69 & 420
\\\hline
Python & 105 & 60 & 28 & 33 & 226
\\\hline
\end{tabular}
\end{table}

We present figures for the number of lines of code necessary to create clients in both Go and Python in Table~\ref{tbl:client_code}.
The Python client was easily developed in under one day of effort, indicative of the simple nature of the communication protocol and the ease with which it could be implemented on any number of platforms and devices.
In this regard we consider ourselves highly successful, providing very high availability while managing to require extremely simple client logic.

%%% Local Variables:
%%% mode: latex
%%% TeX-master: "paper"
%%% End:


\section{Future Work}

\subsection{Alternate Designs}
\label{subsec:alternate_designs}

% TODO this could probably be cleaned up a little 

Currently, a full Raft transaction is performed through Etcd on every change to the system state.
While this provided a relatively simple design with strong consistency guarantees, it incurs a rather high latency that does not parallelize due to the serial nature of Raft transactions.
One alternate method to explore would be to use Raft only for leader elections, and have the leader stream events directly to the other coordinators rather than submitting them to the Etcd log.
It may be possible to achieve higher throughput using, for example, a 2-phase commit protocol. 

Currently, the full state of the system is stored only at the coordinator, which can become a scalability bottleneck.
We have considered one alternate design which is essentially the opposite of this, with the coordinator storing no system state beyond the set of connected brokers. 
On inbound queries and metadata changes, a broker would broadcast to all other brokers, allowing them to evaluate the changes and set up new forwarding routes as necessary.
This is unfortunately expensive as it requires a broadcast, but it does away with the necessity for replication via Etcd since brokers can recreate the state of their system via information they receive when clients reconnect to them, which may be a desirable tradeoff.

Another option we considered that provides a tradeoff between our current design and the one described above would be to have each broker store only its own state, and have the coordinator store only some sort of heuristic data.
A broker would forward messages to the coordinator, which would not contain the full system state, but have enough information to narrow the possibly affected brokers to a smaller subset as opposed to having to broadcast to the entire system.
This could, perhaps, mean storing ranges of metadata values that publishers at brokers contain.
This pushes off the query processing effort onto the brokers and avoids full system broadcasts, making it scalable, but unfortunately still requires the coordinator to have a consistent view of the system to avoid the situation where the coordinator doesn't forward a message to a broker which it should have.
However, as this view of the system is only required for performance rather than correctness (since all messages could still be broadcast), it may prove to be a desirable point in the design space.


\section{Conclusion}

\bibliographystyle{IEEEtran}
\bibliography{IEEEabrv,references}

\end{document}



%%% Local Variables:
%%% mode: latex
%%% TeX-master: t
%%% End:
