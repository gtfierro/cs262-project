\section{Related Work}
\label{section:relatedwork}

The CQBS distributed broker is most closely related to \emph{publish-subscribe} systems.
We examine each of the systems below along the primary design dimensions of the CQBS broker: richness of publisher descriptions, expressiveness of syndication model, client complexity and fault tolerance.

\begin{table*}
\caption{High-level comparison of features between systems}
\label{table:comparison}
\centering
\begin{tabular}{|l|c|c|c|c|c|}
\hline
\textbf{System Name} & \textbf{Publisher Descriptions} & \textbf{Syndication} & \textbf{Client Complexity} & \textbf{Client Failover} & \textbf{Fault Tolerant} \\
\hline \hline
redis~\cite{redis} & $N$ channels & List of channel names & Simple & None & Replicated cluster \\
MQTT~\cite{locke2010mq}\cite{hunkeler2008mqtt} & Hierarchical topics & Wildcard matching & Simple & None & None \\
XMPP~\cite{saint2011extensible} & Unique names & List of publishers & Complex & None & Federated Servers \\
Kafka~\cite{kreps2011kafka} & Hierarchical topics & Wildcard matching & Complex & Yes & Replicated broker cluster \\
SIENA~\cite{carzaniga2000achieving} & attribute-value pairs & SQL-like predicate & N/A & N/A & Replicated broker, flexible routing \\
JMS~\cite{hapner2002java} & attribute-value pairs & SQL-like predicate & Complex & Unknown & Yes \\
CORBA~\cite{vinoski1997corba} & properties and names & property matching & Complex & No & No \\
\textbf{CQBS Broker} & attribute-value pairs & SQL-like predicate & Simple & Yes & Replicated brokers, coordinators \\
\hline
\end{tabular}
\end{table*}

\subsection{Topic-Based Publish-Subscribe}

Most publish-subscribe (``pubsub'') systems fall into one of two categories: topic-based and content based~\cite{eugster2003many}.
The most basic form of topic-based pubsub is a channel model, in which producers (data publishers) transmit data associated with some channel name to a broker; subscribers list the channels in which they are interested.
The benefits of this approach are its simplicity and speed---the ``hot path'' of a published message simply retrieves a list of subscribers---but there is limited expressive power because each publisher can only describe their data using a single dimension (the name of the channel)~\cite{redis}.
Because of these limitations, some modern pubsub systems use hierarchical topics with prefix and suffix matching using wildcards. The most popular of these are MQTT~\cite{locke2010mq}, Kafka~\cite{kreps2011kafka} and XMPP~\cite{saint2011extensible}.
%While our CQBS system is primarily a \emph{publish-subscribe} solution, its goals of providing
%resource discovery as well as data transfer means that we must also

%We examine each of the systems below along the primary design dimensions of
%the CQBS broker: richness of publisher descriptions, expressiveness of
%syndication model, client complexity and fault tolerance.

\textbf{MQTT} is a lightweight publish-subscribe protocol popular for its simpicity and extensible messages.
Producers of data in MQTT publish on hierarchical path-like topics such as \texttt{/apartment/gabe/livingroom/temperature}.
This construction of topics is best for grouping publishers together along a limited number of dimensions, but quickly becomes unwieldy as the dimensionality and sparsity of the descriptions increase.
For example, a temperature sensor could be described along the following dimensions:

\begin{itemize}
\item manufacturer and model number
\item city, campus, building, floor and  room number
\item orientation or position within a room
\item accuracy and precision of temperature sensor
\item method of temperature sensing (e.g. IR, thermopile)
\item who installed the temperature sensor and when it was installed
\item sample rate of the temperature sensor
\end{itemize}

To be effective, a topic-based system must determine the order and syntax of topics so that subscribers can know they are consuming the appropriate streams.
This is further complicated when considering other types of publishers which may include an entirely different set of descriptive tags.
MQTT syndication supports prefix matching on topics and limited forms of suffix matching.
To subscribe, applications specify explicit topics (\texttt{a/b/c/d}), one-level wildcards (\texttt{+/b/c/d, a/+/c/d, a/+/+/d, a/b/c/+}) and multi-level wildcards (\texttt{\#, a/\#, +/b/c/\#}).
While appropriate for basic subscriptions, this approach does not allow the expression of more complex predicates that contain ``and'', ``or'' or ``not'' relations: temperature sensors in Gabe's apartment, but not the ones in the kitchen or the bedroom.

MQTT supports three Quality-of-Service levels for message delivery: at most once, at least once, and exactly once.
Most client implementations simply address the first two, keeping complexity and code-size down; it is entirely feasible to implement a MQTT client on an embedded device, and there exists an adaptation of MQTT (MQTT-S~\cite{hunkeler2008mqtt}) for non-TCP/IP networks.
MQTT does not contain any explicit fault tolerance mechanisms: brokers may be distributed and ``bridged'' with the aid of an administrator, but failover logic is entirely implementation dependent.
In summary, MQTT is insufficient for our goals because hierarchical topics are fundamentally constraining in their structure and syndication.

\textbf{XMPP}~\cite{saint2011extensible}, or the Extensible Messaging and Presence Protocol, is an XML-based technology for instant messaging, video conferencing and more recently sensor communication (as seen in large, deployed systems such as Sensor Andrew~\cite{rowe2011sensor}).
Similar to channel-based systems, every entity (publisher or subscriber or broker) in a distribution of federated XMPP servers has some unique address from which it can send and receive messages.
There are attempts to provide a more expressive pubsub model on top of XMPP that allows for the discovery of services and subsequent subscription to relevant service providers~\cite{millard2010xep}.
These service descriptions can be extended to include arbitrary contextual data, which is a definite advantage over primitive address-based messaging.

The limitation of XMPP is in its size and complexity.
XMPP messages are XML-encoded and thus permit the expression of many different structures, but XML parsers are typically large and memory-intensive, rendering them inappropriate for embedded devices.
Additionally, XML tends towards large messages, which can result in fragmented messages in embedded networks that lower throughput and delivery rate.

\textbf{Apache Kafka}~\cite{kreps2011kafka} is a distributed messaging system designed for data pipelining in large, distributed, high-throughput applications.
Kafka is not designed for deployment scenarios that require rich descriptions of the array of available data services, so publisher and subscriber interactions are done via hierarchical topics and wildcard matching (very similar to MQTT).

Kafka, unlike MQTT and XMPP, is designed to be fault tolerant: all topics are replicated in a cluster of brokers, and failover is automatic.
To achieve the combination of fault tolerance and performance, Kafka clients must be carefully engineered, and are intended to be fuller applications rather than embedded devices.
Thus, while Kafka may be suitable for a data anaylsis pipeline in a datacenter, it does not meet our requirements for message delivery and reception at the ``edge'' of an IoT network.

\subsection{Content-Based Publish-Subscribe}

In content-based pubsub systems, publishers attach richer descriptions to messages, allowing subscribers to specify predicates that act as filters for which messages they receive.
While this scheme has richer descriptive power, routing on a per-message basis is computationally expensive (reducing routing efficiency) and can require larger messages from publishers.
% these will be fairly short

\textbf{SIENA}~\cite{carzaniga2000achieving}, the Scalable Internet Event Notification Architecture, aims to maximize the expressiveness of publishers and subscribers communicating in a wide-area network.
SIENA publishers send messages containing a set of \texttt{<type, name, value>} tuples, to which clients can subscribe using SQL-like filter expressions.
This approach, designed to provide discovery of relevant data in a large amount of heterogeneous messages, allows publishers to express richer metadata than would be feasible using a topic-based scheme.
It also allows subscribers to more precisely define the data they need.
For example, this could easily capture the proposed descriptive elements of the temperature sensor described above and allow a subscriber to filter on any combination of those attributes.

SIENA is fully distributed, using a tree overlay for routing and a special ``merging'' mechanism for pruning unnecessary delivery of messages to subtrees.
The disadvantages of SIENA mirror those of other content-based systems: carrying a full description of a publisher with every message reduces the bytes available for application data in embedded, constrained networks typical of the IoT.

\textbf{JMS}~\cite{hapner2002java}, the Java Message Service, is a distributed messaging service for connecting distributed application components.
Publishers send messages with headers containing standardized key-value pairs, but also contain lists of user-defined key-value properties.
Like SIENA, JMS clients subscribe with SQL-like expressions that express constraints on the set of publisher properties.
One difference between JMS and other systems is its default setting of exactly-once delivery, which complicates client logic and raises the network overhead of sending or receiving a message.

Distribution and fault-tolerance in JMS is possible, but requires specific configuration of which topics are distributed and among which servers they are distributed.
The focus of JMS is on enterprise-type applications that are in need of a messaging system that can adapt to its needs, but does not need to do so dynamically.

%\textbf{Elvin}

%\subsection{Other Systems}

%\textbf{Tuple Spaces} \todo{tuple spaces: t-spaces and linda}

\textbf{CORBA}~\cite{vinoski1997corba}, the Common Object Request Broker Architecture, is a data bus for communication amongst distributed objects that provides property- and name-based discovery and event notification.
Distributed object models (including DCOM~\cite{horstmann1997dcom}) share many features with content-based pubsub systems: similar to JMS, published messages contain key-value pairs in the header and body, and syndication is performed by subscribers specifying filters on those attributes.
CORBA itself is limited because it is not designed to be distributed, and has no failover or replication mechanisms.

%%% Local Variables:
%%% mode: latex
%%% TeX-master: "paper"
%%% End:
