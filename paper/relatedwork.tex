\section{Related Work}

%TODO: do we define/justify the basic principles of pubsub anywhere here, e.g. decoupling between publishers and subscribers?

The CQBS distributed broker is most closely related to \emph{publish-subscribe} systems, but because of its intended use in building dynamic applications for the so-called ``Internet of Things'', we also draw comparisons between it and several IoT application frameworks.
We examine each of the systems below along the primary design dimensions of the CQBS broker: richness of publisher descriptions, expressiveness of syndication model, client complexity and fault tolerance.

\begin{table*}
\caption{High-level comparison of features between systems}
\label{table:comparison}
\centering
\begin{tabular}{|l|c|c|c|c|c|}
\hline
\textbf{System Name} & \textbf{Publisher Descriptions} & \textbf{Syndication} & \textbf{Client Complexity} & \textbf{Client Failover} & \textbf{Fault Tolerant} \\
\hline \hline
redis~\cite{redis} & $N$ channels & List of channel names & Simple & None & Replicated cluster \\
MQTT~\cite{locke2010mq}\cite{hunkeler2008mqtt} & Hierarchical topics & Wildcard matching & Simple & None & None \\
\hline
\end{tabular}
\end{table*}

\subsection{Topic-Based Publish-Subscribe}

Most publish-subscribe (``pub-sub'')systems fall into one of two categories: topic-based and content based \cite{eugster2003many}.
The most basic form of topic-based pub-sub is a channel model, in which producers (data publishers) transmit data associated with some channel name to a broker; subscribers list the channels in which they are interested.
The benefits of this approach are the its simplicity and speed -- the ``hot path'' of a published messages simply retrieves a list of subscribers -- but there is limited expressive power because each publisher can only describe their data using a single dimension (the name of the channel)~\cite{redis}.
Because of these limitations, some modern pub-sub systems use hierarchical topics with prefix and suffix matching using wildcards. The most popular of these are MQTT~\cite{locke2010mq}, Kafka~\cite{kreps2011kafka} and XMPP~\cite{saint2011extensible}.
%While our CQBS system is primarily a \emph{publish-subscribe} solution, its goals of providing
%resource discovery as well as data transfer means that we must also

%We examine each of the systems below along the primary design dimensions of
%the CQBS broker: richness of publisher descriptions, expressiveness of
%syndication model, client complexity and fault tolerance.

\textbf{MQTT}
\subsubsection{Publishing} Producers of data in MQTT publish on hierarchical path-like topics such as \texttt{/apartment/gabe/livingroom/temperature}.
This construction of topics is best for grouping publishers together along a limited number of dimensions, but quickly becomes unwieldy as the dimensionality and sparsity of the descriptions increase.
For example, a temperature sensor could be described along the following dimensions:

\begin{itemize}
\item manufacturer and model number
\item city, campus, building, floor and  room number
\item orientation or position within a room
\item accuracy and precision of temperature sensor
\item method of temperature sensing (e.g. IR, thermopile)
\item who installed the temperature sensor and when it was installed
\item sample rate of the temperature sensor
\end{itemize}

To be effective, a topic-based system must determine the order and syntax of topics so that subscribers can know they are consuming the appropriate streams.
This is further complicated when considering other types of publishers which may include an entirely different set of descriptive tags.

\subsubsection{Subscribing} MQTT syndication supports prefix matching on topics and limited forms of suffix matching.

\subsubsection{Client Complexity}

\subsubsection{Fault Tolerance}



here we talk about mqtt
