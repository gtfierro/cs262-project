\documentclass{article}
\usepackage[utf8]{inputenc}
\usepackage{natbib}
\usepackage[in]{fullpage}
\usepackage{filecontents}

\title{CS262 Proposal}
\author{Gabe Fierro}
\date{February 2016}

\begin{document}

\maketitle

\section{Problem Statement and Motivation}

The context of the Internet of Things has seen an increase in both the number
and capabilities of small, low-powered, constrained devices wanting to interact
with each other and the outside world.  This has raised the question of how to
conduct discovery and communication. The publish-subscribe (\emph{pub-sub}) is
an attractive approach because it decouples publishers from subscribers in
space (communication through a well-known broker helps deal with firewalls and
NATs), and reduces load on popular publishers.

Ensembles of networked ``things'' interacting with dynamic applications require
rich descriptive power to promote discovery across heterogeneous devices and
services, which should be reflected in the syndication mechanism. 

as well as the ability to quickly react to changes in the set of
subscribed sources.

There are two dominant ``flavors'' of pub-sub -- topic-based and content-based
\footnote{\cite{eugster2003many} actually names type-based pub-sub as a third
type, but it is very smilar to content-based} -- that traditionally identify
tradeoffs between performance and expressiveness. In topic-based pub-sub
systems, messages are published to logical channels that may be in a flat or
hierarchical namespace, and subscribers identify a name or ``glob'' that
matches topics. The benefits are that matching is typically fast and message
overhead is small, but the expressive power of a ``topic'' is limited.  In
content-based pub-sub, subscribers specify predicates, which act as filters for
incoming messages for publishers. While this scheme has richer descriptive
power, routing on a per-message basis is computationally expensive (reducing
routing efficiency) and can require larger messages from publishers (if
messages must contain descriptive elements for routing).


needs to provide for the low-bandwidth and duty-cycled nature of embedded motes,
high expressisve and descriptive power to enable dynamic applications and low latency
response times for real-time clients.

\section{Prior Work}

\if 0
list of topic based
kafka
mqtt

list of content based
rabbitmq
\fi


\section{Estimate of Results and Deliverables}

\bibliographystyle{abbrv}
\bibliography{proposal}

\begin{filecontents}{proposal.bib}
@article{eugster2003many,
  title={The many faces of publish/subscribe},
  author={Eugster, Patrick Th and Felber, Pascal A and Guerraoui, Rachid and Kermarrec, Anne-Marie},
  journal={ACM Computing Surveys (CSUR)},
  volume={35},
  number={2},
  pages={114--131},
  year={2003},
  publisher={ACM}
}
\end{filecontents}


\end{document}

\if 0
# Problem Domain + Motivation

# Description of what i am going to do and what is new about it

# how I am going to do the evaluation
- client complexity/size
- throughput
- latency of re-evaluation
- descriptive power? similar to how arkas paper was, how well do 'capabilities' cover the same object as 'interfaces'?

# list all the resources i need

# list all of the participants :(

\fi
